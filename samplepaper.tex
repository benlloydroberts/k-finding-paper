% This is samplepaper.tex, a sample chapter demonstrating the
% LLNCS macro package for Springer Computer Science proceedings;
% Version 2.20 of 2017/10/04
%
\documentclass[runningheads]{llncs}
%
\usepackage{graphicx}
\usepackage{amsmath}
% Used for displaying a sample figure. If possible, figure files should
% be included in EPS format.
%
% If you use the hyperref package, please uncomment the following line
% to display URLs in blue roman font according to Springer's eBook style:
% \renewcommand\UrlFont{\color{blue}\rmfamily}

\begin{document}
%
\title{Contribution Title\thanks{Supported by organization x.}}
%
%\titlerunning{Abbreviated paper title}
% If the paper title is too long for the running head, you can set
% an abbreviated paper title here
%
\author{Ben Lloyd-Roberts\inst{1} \and
Phil James\inst{1} \and
Michael Edwards\inst{1} \and 
Tom Werner\inst{2}}
%
\authorrunning{F. Author et al.}
% First names are abbreviated in the running head.
% If there are more than two authors, 'et al.' is used.
%


\institute{Swansea University, Swansea, UK \and 
	Siemens Rail Automation UK, Chippenham, UK \\
\email{\{ben.lloyd-roberts, p.d.james, michael.edwards\}@swansea.ac.uk}\\
\email{WT.Werner@siemens.com}}
%
\maketitle              % typeset the header of the contribution
%
\begin{abstract}
Formal verification is an increasingly popular approach to guaranteeing safety in critical systems. The railway in particular shares a history with formal methods spanning decades[1,2,3,4]. In this thesis we explore the limitations of current verification techniques applied to railway interlockings and how machine learning can be used to support SAT-based model checking in particular. We provide a brief overview of both disciplines and the necessary background knowledge required to replicate our approach. Finallywe present the technical implementation of a reinforcement learning (RL) environment modelling the state space of a ladder logic program responsible for encoding basic safety properties for a pedestrian controlled light crossing. An RL agent is then successfully trained to output two sets if minimal transitions for traversing the state space to aid in identifying loop free paths for further verification.

\keywords{First keyword  \and Second keyword \and Another keyword.}
\end{abstract}
%
%
%
\section{Introduction}
\subsection{Motivations}
Advancements in machine learning (ML) research in recent years have marked an inflection
point in private and public sector investments in predictive modelling. Consequently
the development of autonomous intelligent systems has facilitated an unprecedented
level of its adoption in complex systems. The range of applications are staggering, from
autonomous vehicle navigation through computer vision, to assisted medical diagnostics.
One of three disciplines comprising the field of ML is Reinforcement learning (RL). RL
differs from other data driven ML techniques in its ability to automate tasks without the
dependency of large datasets. In this project we leverage the ability of a software agent
to rapidly learn new and complex tasks with the existing formal verification techniques
applied to railway interlockings. Additionally, we utilise verification approaches for
Ladder Logic developed across several projects carried out at Swansea University in
co-operation with Siemens Rail Automation UK. We aim to explore if approaches within
ML can be used to improve the efficiency of verification whilst also reducing the number
of false negatives that are reported by the current approach of model checking.

The current SAT-based technique used for verification faces certain limitations which this
project intends to address. Due to the scale and complexity of modern critical systems
such as interlockings, contemporary approaches require mathematical abstractions of the
system’s technical implementation. Finite state machines1 are a well-established and robust
modelling technique that can be traced to ‘computing machines’ presented by Turing in
1936 [15]. A system’s state space can be viewed as a graph. Each node representing a unique. configuration of that system where edges between nodes2 denote the transition from one
state to another. It is by verifying each state of the system according to some formal safety
properties that we can guarantee correctness. Even the most basic systems require loops
to perform a predefined number of, or infinitely many, repeated actions. The previously
mentioned approach of model checking faces limitations in this regard. Ultimately our
objective is to determine the depth3 of a given state space from some initial state to identify
a k-number of steps for loop free paths. That is to say we wish to reachability from an
initial state to some kth state such that no transition or state is observed more than once.
We therefore propose the introduction of an RL software agent to sufficiently determine
the maximal depth through exploration and exploitation of the state space and transition
rules. We first formulate the task as an optimisation problem. Initially we define an
environment similar to a state space representation of a simple ladder logic program used
for pelican crossing verification. This environment is then used to train and RL agent to
first, discover all unique states and transitions within that state space and output a trace
for said path4. This can then be used to reduce the search space of loop free paths.


\section{Railway Signalling}
\subsection{Components of Signalling}
Railways have always been hazardous environments. The mass and velocity of locomotives,
coupled with their restricted movement and fixed trajectory presents a risk to passengers
and staff alike. As a result, accidents occurred frequently in the early days of railway
operation. Risks of head on collisions were exacerbated by single lane tracks, requiring
the notion of movement authority and a set of rules to safely direct railway traffic. Thus
the principles and practice of signalling were introduced. Movement authority can be
granted or withdrawn based on the current state of the track to ensure that trains proceed
safely, without risk of collision. Signalling on the other hand determines how and when
movement authority is changed. We now introduce some basic terminology to describe
the composition of railway networks1 in order to understand the principles of signalling.
In this section we briefly discuss the function of four key railway components; signals,
track segments, routes and points.
To grant or remove movement authority necessitates a system capable of informing
train operators of such changes. Thus physical signals were introduced. Initially human
operators, referred to as policemen, monitored railway activity using a stopwatch to time
departures and hand signals to notify train drivers2. The meaning behind each signal is indicated by the signal’s aspect3. Traditionally fixed signals placed near track edges had
physical aspects in the form of semaphore signals. Most modern signals are electronic,
providing visual feedback in the form of coloured lights, denoted by the prefix s in Fig.
2.1. Common aspects are denoted by three colours; Red indicating the driver should halt,
yellow suggesting drivers should proceed with caution and green to indicate a clear track
ahead.
In order to track train positions railway networks are built up of smaller track seg-
ments, denoted by the prefix t in Fig. 2.1. Each segment is fitted with an electronic
circuit to indicate whether it is occupied by a train or not. Signals are usually placed
at block intervals along these track segments to prevent any two trains from occupying
the same section. Consequently engineers can be notified of any future obstructions
before they proceed to the affected region. Each track segment, by definition, requires a
clear start and end point. They do not necessarily construct a linear track and will often
lead to a split or junction in the network. Sections where track segments lead to such
deviations are referred to as points, denoted by the prefix p in Fig. 2.1. Points are typic-
ally lever operated sections of track regulated by a nearby computerised or human controller.
Finally we introduce the idea of routes, the physical path that a train may follow
along the railway. Route setting can be performed manually by a human operator or
automatically by a computerised system. The potential routes available to a train are then
subject to the number of points and their positions. Routes are then set by altering signal
aspects and points along the necessary track segments while also considering the routes
of other trains occupying the network. It is clear to see that route setting is a critical stage
of railway operations and necessitates some form of verification. Here we introduce the
notion of an interlocking.

\subsection{Interlockings}
The term interlocking describes the system used to prevent safety property violations in
railway signalling. Interlockings serve as a filter or ‘safety layer’ between inputs from operators, ensuring any changes made to the current railway state does not lead to an
unsafe state. That is to say any route setting or track changes must be verified by the
interlocking system before any physical change takes place. It is also the interlocking’s
responsibility to prevent any attempts to force unsafe actions. In the event a fault is detected
during routine operation the interlocking should initiate a fail safe state. Typically this
means setting all signal aspects to red and bringing traffic to a halt until the issue is resolved.
The original 19th century interlocking systems comprised of mechanical levers for
controlling railway components and a locking bed to prevent operations conflicting
with safety conditions. As railway traffic increased so did operational complexity and
risk. Fortunately by the 1930s relay interlockings had superseded mechanical levers,
introducing buttons and electrical circuits to both monitor train positions and partially
automate the signalling process. In the 1980s significant legislation was passed allowing
the embedding of microprocessors in safety-critical systems. Solid state interlockings (SSIs)
were developed shortly after. Contemporary interlocking systems are largely software
controlled, offer greater flexibility for maintenance and support automatic route setting
through computerised controllers.
Such levels of complex automation require interlockings to process large sets of data
from varying sources before altering the state of the railway. Embedded systems within
trains, track circuits and other sensor arrays transmit data back to the interlocking pertaining to the current state of the railway network. Every mutable component must be considered precisely when computing subsequent states of a changing railway network.
This is particularly challenging given computerised railways operate as immensely complex
distributed systems and therefore are subject to concurrency issues. As such, interlockings
require a robust method for validating and verifying routes. Consequently interlockings
have also become increasingly complex, presenting new challenges in terms of how they
perform verification.
Interlocking systems require accurate information in real-time to monitor the current
state of the railway network4. First the interlocking must read input values from a range of
various sources. These values often pertain to the mutable railway components discussed
earlier. Inputs include but are not limited to; operational requests sent by the signaller,
train positions detected by track circuits or aspects currently displayed by an array of
signals. The recurrent nature of interlockings also allow output values from previous
cycles to be used as inputs for subsequent executions.
Once the necessary inputs have been received by the interlocking they are used to
compute output values to pass through the ladder logic program. For now readers should
consider this stage as the layer responsible for ensuring proposed changes to the railway
composition comply with a strict set of safety properties. The specific functions and
semantics of ladder logic programs are discussed in section 2.3.
Finally outputs verified by the ladder logic program are committed back to the initial
range of input sources. Here any safe track changes can be made. Additionally the signaller
is sent an update from the interlocking providing information regarding the latest control
cycle. This process repeats indefinitely until the fault detection system identifies a problem.

\subsection{Ladder Logic Programs}
In the early days of process control, engineers required a way of implementing electrical
circuits that could behave logically. Initially this was achieved using racks of electrical
relays implemented with combinational logic. Schematic diagrams were then used to
represent various relay devices and their physical connections. Such schematics were designed and documented using ladder logic to represent the circuit structure. Ladder logic
has since evolved into a graphical programming language widely used in programmable
logic controllers (PLCs). Program diagrams are comprised of horizontal lines, referred
to as rungs, and logical connectors between symbols, known as coils and contacts. Two
types of contacts are used to represent the value of a contained program variable, also
referred to as a latch. Open contacts as shown in Fig. 2.2(a) represent the unchanged
value of a latch while closed contacts denote their negated value, Fig. 2.2(b). Latches
often represent physical inputs to the system but are also used for any other variables
required. Coils can represent one of two output types. These are either physical outputs
for some system component or a computed value to be used later in the program. Output
values are also referred to as latches and can be negated with a closed coil. Each rung
is read from left to right and, using the logical connectives shown in Fig. 2.3, represent
a condition or rule within the program. Rungs are closed off with a coil representing
conditional output on the right hand side. Using coils and contacts enables the expression of atomic propositions in our programs.
However we still require a way to represent connectives. Fig 2.3 shows that two types of
connections can be established between contacts. Horizontal lines denote conjunction, equivalent to a logical AND connective in propositional logic5. Vertical lines between
contacts denote disjunction, equivalent to a logical OR connective in propositional logic.

\section{Verification and the Railway}
As discussed in our introduction, formal verification of large systems is often infeasible without abstractions since there are too many variables to consider at once. However abstraction can often result in a trade-off between maximising coverage and capturing sufficient detail of the target system. We thus introduce the notion of validation vs verification.
Safety properties for example are derived through a series of different processes; first in a requirements document expressed in natural language. This articulates what the system should be capable of but not necessarily how it achieves this. The implementation of said safety properties is defined in a specification document. These documents are defined in a formal specification language and are not necessarily reflective of the system requirements.
Validation therefore ensures correctness according to the system requirements where verification concerns correctness w.r.t the formal specification. Consequently formal verification can only guarantee correctness if the specification is correct. For railway interlocking, safety properties are defined in control tables expressed in ladder logic.

\section{Mapping Formal Methods to RL}




\section{Reinforcement Learning}
Humans are exceptionally unique in their ability and aptitude to learn through experience.
From gestation to formal education our body and brain learn to respond to their environ-
ment. Exposure to new information and a means of processing that information is how
we learn to function independently. Consider the ability to communicate using natural
language, which is often taken for granted. This is generally learnt through observation,
repetition and correction. ML is conceptually similar in this regard. ML is the theory and
practice of utilising large collections of related information, or datasets, to identify and
predict patterns. Using intelligently designed algorithms, ML allows us to generalise the
behaviour required to perform a given task autonomously. Fig. 4.1 illustrates a high level
overview of the main topics within ML. The three primary approaches are supervised,
unsupervised and reinforcement learning (RL). Each approach differs from the other based
on the algorithms they employ and chiefly, how observations are drawn from existing data.
Selecting the appropriate methodology is heavily influenced by the application domain.
In this section we provide a general overview of ML theory and practice, primarily in
supervised learning given its relevance to our work. In supervised learning our objective is to learn a general function which maps a given
input x to the correct output y. This learning approach differs from its counterparts by
utilising training labels, see Fig. 4.2. Labels are typically represented by a vector, Y ,
containing discrete values associating inputs from a training set X to their correct outputs.
Ultimately our goal is to train a sufficiently robust model to understanding the underlying
structure of the data distribution. Consequently the model is able to take observations from
previously unseen data and predict the correct output without knowledge of the data label.Learning from past experiences is a capability of most intelligent life. Humans in particular
are exceptional in this regard. The rate at which we process information, reason logically
and derive conclusions based on what we can observe is exceptional compared to other
animals. We do however, face certain biological and physical limitations that influence our
rate of thinking. Contrarily, computers have no ability to operate independently without a
set of explicit instructions. Yet, recent advances in ML have proven machines can not only
excel at complex tasks [41, 42, 43] but learn to match, if not surpass, human performance
levels [44]. Reinforcement learning (RL) is one such area of ML which leverages human
ability to formulate problems and utilise computational means of solving them through a
process of accelerated trial and error. RL centres around the concept of a software agent
capable of learning optimal policies for interacting with a human specified environment.
In this section we discuss the principles of reinforcement learning in general, common
algorithms and their relevance to this work. 

The above definition provides us with the necessary elements to construct our own
propositional representation of a ladder logic program. First we distinguish between input
and output variables. Each rung provides us with an individual coil and corresponding
latch denoting the conditional output. Unifying all coils in the program under a single set
produces the following definition:


\subsection{Deep Learning}
In traditional ML approaches we are often expected to use a significant amount of
prior knowledge regarding our application domain. Without a clear understanding
of both the data. 

Artificial neural networks (ANN) predate the field of deep learning by several decade.
Originating in work presented by McCulloch and Pitts [40], neural networks are a
computing system modelled after biological networks of neurons found in the nervous
system. Each neuron in the network is represented by a connected node which can
‘fire’ some signal based on activity in a previous layer. A simplistic ANN architecture
can be seen in Fig. 4.7. Ultimately, neural networks are used as universal function
approximators. Given some input we aim to learn the function for a given output.
This allows us to predict future outputs for unseen inputs since we have a general
approximation of their relationship.
Application domains have a significant influence on the NN architecture used. We
observe a number of layers and nodes in Fig. 4.7, forming the network. Different types of
layers adjust the transformation applied data received by each neuron. Fig. 4.7 shows
a dense, fully connected hidden layer which simply connects the output of one neuron
to another 2. Similarly a convolutional layer is a specific type used in image processing.
This however, is beyond the scope of our work. Input layers are how we present initial
information to the network, usually samples from a dataset. Using our earlier example for classification, an input layer would consist of 12 neurons, each representing a feature
(single boolean) of the current state. Relationships between neurons are denoted by a
weighted edge which is assigned some value between 0 and 1. The total weighted sum of
all edges connected to a single neuron are subsequently passed to an activation function.
Typically this is some non-linear function which maps the weighted sum to another value
between 0 and 1. In deep learning this process is continued for every layer until the output
layer is reached. The configuration of the output layer depends entirely on the task we are
trying to learn. For a binary classification problem the output layer would consist of two
neurons, valid and invalid according to our earlier example. The exact value output by the
network will be some probability of the output neuron resembling the ground truth.

As discussed in the previous section, Q-tables scale poorly to large environments with
many actions. Deep reinforcement learning (DRL) addresses the limitations of Q-functions
in determining an optimal policy. Instead a neural network, called the policy network, can
be implemented to approximate the optimal Q-function. This process is known as deep
Q-learning where our policy network is a deep Q-network (DQN) [46]. The DQN receives
a number of states from the environment as input to approximate a function generating the
optimal Q-value such that it satisfies the Bellman equation, see Fig. 5.3 Q-values output
by the network are then compared to the target optimum to compute the loss. Ideal for
approximation, the deep neural network architecture will then learn do minimise this loss
by adjusting weights using SGD and back propagation. This process is repeated for all
state-action pairs until the optimal Q-function has been approximated.

To support the training of our policy network we generate a dataset of experiences
sampled at each time step. A value N is set to limit the size of our dataset, referred
to as replay memory. Using training experiences from replay memory helps avoid learning local trends in
the data, see Fig. 5.4. Highlighted data points represent experiences randomly sampled
for training, known as a minibatch. Note how the larger replay memory provides more
general coverage of past experiences meaning linear correlations in short sequences are
less influential on learning. Our first training cycle involves setting a replay memory size
N and random weight initialisation for the policy network. Any state preprocessing is then
performed before being passed to the DQN. We compute the loss for the given Q-value
output by the network for the action listed in our most recent sample from replay memory.
In order to compute the loss for the current training iteration, we require some method for
calculating the maximum expected discounted reward for the next state. Without a Q-table
storing a comparable value for s$^{\prime}$ we must approximate this value by passing s$^{\prime}$ to the
policy network using different samples from replay memory. Thus we can determine the
greatest Q-value output by the network over all actions a$^{\prime}$1,...,a$^{\prime}$n, in s$^{\prime}$. Next we perform
gradient descent, typically stochastic gradient descent (SGD), to minimise the loss and
update the weights of our DQN. We repeat this process for every new time step until an
optimal policy is found. Alternatively we incorporate a second DQN, called the target network. This aims to
solve the problem of approximating future maximum expected reward with a single policy
network. Given weights are only adjusted once the loss has been computed, both q(s,a)
and q(s$^{\prime}$,a$^{\prime}$) are approximated using the same weights. This introduces undesirable properties for learning. As the network weights adjust, q(s,a) approaches the target value
while our approximation of q$\star(s^{\prime}$,a$^{\prime}$) continues to change. Using a second DQN allows
us to adjust a separate set of weights by for outputting target Q-values. These are then
adjusted at arbitrary time steps to match the current weights of the policy network.

\begin{theorem}
	(Experience Replay): The agent’s experience at a given time step, et is
	defined:
	\begin{equation}
	e_{t} = (st,at,rt+1,st+1)
	\end{equation}
\end{theorem}

\subsection{MDPs}
The essential process behind RL is that of gamification [45, 46, 47]. Imagine a player has
been placed in some game environment with no knowledge regarding their objective or
the scope of the environment. Depending on the task we wish the agent to learn, our
environment is encoded with a rule set which describes performable actions and a reward scheme to either penalise or reinforce agent decisions, see Fig. 5.1. We treat this reward as
an in-game score the agent can see. The action set available to the agent influences new
states of the environment. Agents are then incentivised to explore their environments
through repeated sequences of actions while reacting to each reward. Formally, we define
the environment our agent must traverse as a sequential mathematical framework known
as a Markov decision process (MDP). RL differs from other supervised learning methods in some fundamental yet subtle
ways. Agents, unlike models aim to optimise decision making in order to reach some goal.
Supervised and unsupervised learning typically involve training a statistical model on real
world samples, necessitating a data set for training. RL, through repeated simulation can
produce an agent having learned an optimal policy of which humans would be incapable.
That is to say the agent learns independently without the need for external observations.
RL interpreters issue positive or negative reward depending on agent behaviour.
Action sequences we wish to reward are done so with a positive score. Otherwise the
agent is assigned a negative score. Consequently, agents will attempt to find the optimal
policy to maximise cumulative positive reward and minimise negative rewards. This
can be expressed probabilistically $P(s^{\prime},r|s,a)$. Given the current state s and an action a is performed, what is the probability of a subsequent state $s^{\prime}$ leading to a reward r.
Depending on the learning strategy chosen, agents will attempt to maximise the cumulative
reward, meaning future decisions will be considered at each step. What constitutes the terminal step depends on the environment and how it is imple-
mented. In our case we require some metric of agent performance other than reward
to determine whether the environment should reset. Such problems are referred to as
episodic tasks.1. To this end we introduce a terminal condition which computes the total
number of unique transitions and states to check if the agent has discovered them all.
Throughout the training process it is expected the agent will minimise the number of
necessary transitions to the point an optimal path is learned. Continuous tasks, which
have no terminal step, face the challenge of enumerating future cumulative reward for
infinitely many steps. One solution to dealing with an infinite horizon is discounting.

\subsection{Policy Learning}
A policy is a rule set learned by an agent which determines the next best action. This is determined based on samples from the current probability distribution over any given action yielding a positive reward given the state.

\begin{theorem}(Policy): A probability distribution $\pi(a|s)$ describes the likelihood of action (a) returning a positive reward given environment state (s).
\end{theorem}

\begin{theorem}
	content...(Trajectory): A sequence $\tau$, comprised of the states s0,...,sn, actions
	a0,...,an, and rewards r0,...,rn experienced by an agent following a policy $\pi$
\end{theorem}
The start state s0 is randomly sampled from a start state distribution, Is in our
case. Continuous random sampling of the probability distribution is done to avoid
repetitive behaviour. Initially agents have no understanding of their environment or
which actions lead to the greatest cumulative positive reward. Most initial actions will
result in negative scoring as parameters adjust over time. Agents are expected to perform
poorly at first, particularly with larger action sets given the potential number of sequences
to sample. It is this evolutionary process that allows RL agents to generalise so well
when learning long term goals.
In the context of this project we have two options for representing our environment.
1) Using maximum coverage where the environment consists of all potential system
configurations or 2) As the subset of all reachable states produced by executions of the
pelican crossing program. The first environment consists of 4096 states, each with twelve
directed edges representing the change in value for a single variable. Thus our action
set comprises all propositional variables in V where a single action entails negating
one boolean. The second environment consists of six states, those reachable by the
system. Each state in this environment changes based on the value of input variable
pressed, resulting in an action set {0,1}.

\begin{theorem}
	(Optimal Policy): iven a finite MDP, there exists at least one optimal
	policy J, over a set of parameters $\theta$, such that:
	$J(\theta) = E\pi[r(\tau)]$
	Finding the optimal policy requires some measure of decision quality. These are
	referred to as value functions
\end{theorem}

\subsection{Value Functions}
Value functions return an expected reward, which determines the overall benefit of a given
policy. They can be differentiated by the observations they use to determine the value. So called state-value functions, denoted v$\pi$, provides the value assigned to a state under
policy $\pi$. Informally, state-value functions produce a valuation assigned to the current
state which determines the expected reward from subsequently adhering to the same
policy $\pi$. We define the value function v$\pi$(s):

% Equations 5.1, 5.2

\begin{theorem}
	(Optimal state-value function): An optimal policy J($\theta$) has an optimal
	state-value function for all s $\in$ S, defined:
	
	\begin{equation}
		v\star(s) = max\pi v\pi(s)
	\end{equation}
\end{theorem}
Informally the optimal state-value function returns the maximum possible expected
return of any policy for each state. Additionally action-value functions, denoted q$\pi$,
provides the value assigned to an action under policy $\pi$. Action-value functions, also
knowns as Q-functions, produce a quality valuation assigned to the current action which
determines the expected reward from subsequently adhering to the same policy $\pi$. We
define the value function q$\pi$(s):
 % Equations 5.4, 5.5
 
The output of our Q-function is known as the Q-value.

\begin{theorem}
An optimal policy, J$\theta$ has an optimal Q-function for
all s $\in$ S and all a $\in$A:
\begin{equation}
	Q\star(s,a) = max\pi Q\pi(s,a)
\end{equation} 
\end{theorem}

Informally, the optimal Q-function returns the maximum possible expected return
of any policy for all state-action pairs. The Bellman principle of optimality [48], Eq. 5.7,
states for any state-action pair at the current time step, the expected return from an initial
state s, taking action a and following the optimal policy J($\theta$) thereafter is equal to the
expected reward from taking action a in state s, plus the maximum achievable expected
discounted return from any subsequent state-action pairs.

The Bellman optimality equation is an integral metric used to learn the optimal Q-function
which in turn is used to learn the optimal policy. Given an optimal Q-function action a$^{prime}$, a
Q-learning algorithm will find the best action a$^{\prime}$ which maximises the Q-value for s$^{\prime}$.

% Equation 5.7

\subsection{Reward/Goal Shaping}
Q-learning refers to a policy learning method which uses Q-functions to calculate maximum
expected future reward. Learning is formulated as an iterative process of parameter
adjustment known as value iteration.

\subsection{Exploration Strategy}
In the previous section we briefly discussed initial exploration rates to compensate for
zeroed Q-tables. A popular approach to setting exploration rates is the epsilon greedy
strategy [49, 50, 51, 52]. In order to balance the exploration-exploitation trade-off [53, 54],
we set an exploration rate $\epsilon$ between 0 and 1 to dictate whether the agent prioritises
exploratory or exploitative behaviour. Values nearing 0 represent a greedy strategy where
the agent is more likely to exploit previous knowledge. Epsilon values nearing 1 therefore
encourage exploratory behaviour. Additionally we assign an epsilon decay value to
decrease the exploration rate for each episode. Including an epsilon decay value provides
some beneficial learning properties. As the agent learns more about the environment it
relies less on exploration. Instead we expect to have observed a sufficiently comprehensive
understanding of the environment thus settle on a purely greedy strategy. We also
influence exploration and exploitation at every time step. First we randomly generate
a number, i between 0 and 1 to compare with the current value of $\epsilon$. If i > $\epsilon$ the agent
selects an action with the highest Q-value for the state-action pair. Where i < $\epsilon$, the agent
randomly samples an action to explore the environment.
In this project we aim to train an RL agent to learn the optimal path through our
pelican crossing state space. Say we issue a positive reward of 1 for every new state
discovered and -1 for repeated transitions. To update the new Q-value we approximate
the right hand ride of the Bellman equation. First, we iteratively compare the loss of
the current Q-value and the optimal Q-value for each state-action pair. Our aim is to
minimise this loss until convergence.

% Equations 5.8, 5.9

Designing appropriate reward systems is arguably one of the greatest challenges in RL. Reward shaping is often performed specifically for the application

% Definition 5.10


\section{Implementation}
\subsection{LL program generator}
\subsection{Learning Environment}
Constructing an environment is arguable the longest phase of training an RL agent. To
ensure our agent performs well in real-world applications, the learning environment must
be sufficiently representative of the problem domain. Through what is essentially the
process of gamification, we design an environment that records agent actions and issues a
reward based on the behaviour we wish to reinforce. This process presents a number of
distinct challenges in our context. First, given a simple ladder logic program, is it possible
to encode its functionality and constraints using an imperative programming language.
Second, an efficient method of indexing visited states during exploration to determine a
k-number of steps before the agent is forced to revisit states. Finally, we must identify the
optimal reward scheme and learning parameters to enforce the desired behaviour. We design an openai-gym [102] like environment to train our agent (Appendix A).
Consider the environment as a game. The agent makes successive transitions between
states from some arbitrary start position sampled from Is. Recall the set I  defined our physical inputs received by the ladder logic program shown in sect. 3.2. At each step the
agent is presented with a binary decision to make, selected from our action set A, where: $A \equiv I$

Our single input variable pressed is valued either 0 or 1. Once an action is selected, each
transition must then be computed as a single execution of the ladder logic program. We
define a transition function, Alg. 1, which receives an action ai, and the current state as
input. Depending on whether the agent selects 0 or 1, the function returns a new state
with an updated valuation. These are the transitions $\sigma$ : q $\sum$→ q$^{\prime}$ defined in sect 3.3 w.r.t
finite state automata. Under the assignment rules of our transition function, the agent will
eventually discover all six reachable states. Fig. 8.1 represents the finished environment
with all paths the agent can explore. Note this is essentially a more complete version of
our automaton in sect. 3.3. The formulation of our environment means the agent benefits
from having to traverse a directed graph. This can often reduce the number of decisions to
consider at each step particularly for large graphs with many edges. Additionally our agent
samples from a small action set, meaning fewer potential transitions from a single node. With our transition function generating a total of six states, we continue to implement
a collection of traces to record agent exploration. With each action we store the ensuing
transition, a 3-tuple (q,$a_{i}$,q$^{\prime}$), and the new state q$^{\prime}$ in separate hash tables. Our traces let us determine whether the current state or latest transition is unique given we have
a record of past steps.



\subsection{Agent Training}
Given we wish to find the maximal depth of the state space without repeating transitions,
a positive reward is issued for new discoveries. Contrarily we issue a negative reward
for adding previously recorded states with a harsher reward for repeating transitions. In
regard to the exploration-exploitation trade-off we issue a larger reward for discovering
new transitions over new states. This is primarily due to agent’s prioritising actions that
are guaranteed to issue positive reward while attempting to avoid negative scores. To avoid
infinite loops we introduce a terminal condition which resets the environment when all
states and transitions have been discovered. To encourage the agent to satisfy the terminal
condition as soon as possible we issue a large positive reward once it has been reached.

We can use
DQN(s) to approximate the optimal Q-function and, consequently the optimal policy. o this end we implement two DQNs, the first policy network to train our agent
and the second target network to approximate our target values. DQNs train differently
to Q-tables in their use of replay memory. We implement a basic Replay class to store
sequences of 5-tuples (s,a,s$^{\prime}$,r), where s is the current state, a denotes the action taken,
s$^{\prime}$ representing the next state and r being the reward. We then define a max capacity N,
which dictates the number of experiences our agent remembers. One the max capacity
is reached, the agent pops the oldest entry and adds the latest experience to memory.
During the training process our agent samples experiences according to some arbitrary
batch size. If the replay memory is less than this batch size, we cannot sample. Therefore
we introduce some basic condition checking to determine whether sampling is feasible at
any given time step. In regard to the selected exploration strategy, $\epsilon$, we employ the same
implementation used for Q-tables. Here we set an initial value for $\epsilon$, a minimum value
and our $\epsilon$-decay rate. Concerning our DQN architecture, we construct an input layer of 12
nodes, one for each $v_{i} \in$ V , two fully connected hidden layers and a binary output layer to
reflect the action set A. To avoid the problem of moving targets, we set an update rate of 10
epochs for our target network. That is to say we update the weights of the target network for every 10 passes through our policy network. Our chosen optimiser is Adam [103], an
SGD variant based on AdaGrad [104] and RMSProp [105], due to its speed and accuracy.
\subsubsection{Network Architecture}

\section{Experiments}
\subsection{Extended state-spaces}
The agent learns over time to avoid repeated actions until an optimal trace is output.
This trace can then be used to find the maximal depth of the state space without looping
for all states in Is, if such a path exists. We discover through a series of experiments
that there are two sets of unique transitions for all start states such that no transition is
repeated twice. Concerning the discovery of loop free paths, as discussed in [17], it may
be possible to either explicitly state which path should be verified by the SAT-solver or
leverage discoveries made by the agent in identifying a loop free path. By increasing the
max step threshold within our environment allows the agent sufficient time to explore the
state space and understand what satisfies the terminal condition. Consequently, provided
a long enough training time, the agent will always find an optimal path. In fact it was
discovered through a series of experiments, specifically ones with a low $\epsilon$-decay rate, that
there exist two optimal paths for every start state. These traces can then be analysed to
determine the maximum k-steps before looping.
\subsection{Interlocking examples}





\begin{table}
\caption{Table captions should be placed above the
tables.}\label{tab1}
\begin{tabular}{|l|l|l|}
\hline
Heading level &  Example & Font size and style\\
\hline
Title (centered) &  {\Large\bfseries Lecture Notes} & 14 point, bold\\
1st-level heading &  {\large\bfseries 1 Introduction} & 12 point, bold\\
2nd-level heading & {\bfseries 2.1 Printing Area} & 10 point, bold\\
3rd-level heading & {\bfseries Run-in Heading in Bold.} Text follows & 10 point, bold\\
4th-level heading & {\itshape Lowest Level Heading.} Text follows & 10 point, italic\\
\hline
\end{tabular}
\end{table}


\noindent Displayed equations are centered and set on a separate
line.
\begin{equation}
x + y = z
\end{equation}
Please try to avoid rasterized images for line-art diagrams and
schemas. Whenever possible, use vector graphics instead (see
Fig.~\ref{fig1}).


\begin{theorem}
This is a sample theorem. The run-in heading is set in bold, while
the following text appears in italics. Definitions, lemmas,
propositions, and corollaries are styled the same way.
\end{theorem}
%
% the environments 'definition', 'lemma', 'proposition', 'corollary',
% 'remark', and 'example' are defined in the LLNCS documentclass as well.
%
\begin{proof}
Proofs, examples, and remarks have the initial word in italics,
while the following text appears in normal font.
\end{proof}
For citations of references, we prefer the use of square brackets
and consecutive numbers. Citations using labels or the author/year
convention are also acceptable. The following bibliography provides
a sample reference list with entries for journal
articles~\cite{ref_article1}, an LNCS chapter~\cite{ref_lncs1}, a
book~\cite{ref_book1}, proceedings without editors~\cite{ref_proc1},
and a homepage~\cite{ref_url1}. Multiple citations are grouped
\cite{ref_article1,ref_lncs1,ref_book1},
\cite{ref_article1,ref_book1,ref_proc1,ref_url1}.
%
% ---- Bibliography ----
%
% BibTeX users should specify bibliography style 'splncs04'.
% References will then be sorted and formatted in the correct style.
%
% \bibliographystyle{splncs04}
% \bibliography{mybibliography}
%
\begin{thebibliography}{8}
\bibitem{ref_article1}
Author, F.: Article title. Journal \textbf{2}(5), 99--110 (2016)

\bibitem{ref_lncs1}
Author, F., Author, S.: Title of a proceedings paper. In: Editor,
F., Editor, S. (eds.) CONFERENCE 2016, LNCS, vol. 9999, pp. 1--13.
Springer, Heidelberg (2016). \doi{10.10007/1234567890}

\bibitem{ref_book1}
Author, F., Author, S., Author, T.: Book title. 2nd edn. Publisher,
Location (1999)

\bibitem{ref_proc1}
Author, A.-B.: Contribution title. In: 9th International Proceedings
on Proceedings, pp. 1--2. Publisher, Location (2010)

\bibitem{ref_url1}
LNCS Homepage, \url{http://www.springer.com/lncs}. Last accessed 4
Oct 2017
\end{thebibliography}
\end{document}
