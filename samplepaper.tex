% This is samplepaper.tex, a sample chapter demonstrating the
% LLNCS macro package for Springer Computer Science proceedings;
% Version 2.20 of 2017/10/04
%
\documentclass[runningheads]{llncs}
%
\usepackage{graphicx}
\usepackage{amsmath}
\usepackage{amssymb}
% Used for displaying a sample figure. If possible, figure files should
% be included in EPS format.
%
% If you use the hyperref package, please uncomment the following line
% to display URLs in blue roman font according to Springer's eBook style:
% \renewcommand\UrlFont{\color{blue}\rmfamily}

\begin{document}
%
\title{k-induction bounded model checking\thanks{Supported by Siemens Mobility UK \& EPSRC}}
%
%\titlerunning{Abbreviated paper title}
% If the paper title is too long for the running head, you can set
% an abbreviated paper title here
%
\author{Ben Lloyd-Roberts\inst{1} \and
Phil James\inst{1} \and
Michael Edwards\inst{1} \and 
Tom Werner\inst{2}}
%
\authorrunning{F. Author et al.}
% First names are abbreviated in the running head.,
% If there are more than two authors, 'et al.' is used.
%


\institute{Swansea University, Swansea, UK \and 
	Siemens Rail Automation UK, Chippenham, UK \\
\email{\{ben.lloyd-roberts, p.d.james, michael.edwards\}@swansea.ac.uk}\\
\email{WT.Werner@siemens.com}}
%
\maketitle              % typeset the header of the contribution
%
\begin{abstract}
Solid State interlockings in the railway domain are computerised safety critical systems responsible for operating the signalling components directing railway traffic. The formal verification of such systems using inductive methods, such as bounded model-checking (BMC), often face limited guarantees of correctness given they operate within a bounded region of $k$ depth. In this work we devise a reinforcement learning (RL) strategy to support SAT-based bounded model checking by improving measures of completeness. We outline a mapping from labelled transition systems traditionally used as BMC abstractions, to Markov decision processes (MDP) used as interactive RL environments. Thereafter software agents are trained to explore this environment for the longest acyclic sequences of execution to an upper bound $k$. We test the feasibility of our approach against a series of artificially generated ladder logic programs with incrementally larger state spaces. A number of metrics suggested in RL literature are then gathered to evaluate the robustness and scalability of the approach, particularly in industrial contexts. Finally we produce an \textit{exploration graph} representing the structure of the traversed state space. These visualisations are then fed back to engineers responsible for designing such programs.

\keywords{First keyword \and Second keyword \and Another keyword.}
\end{abstract}


\section{Introduction}
Interlockings serve as a filter or ‘safety layer’ between inputs from operators, such as route setting requests, ensuring proposed changes to the current railway state avoid safety conflicts. As a vital part of any railway signalling system, interlockings are critical systems regarded with the highest safety integrity level
(SIL4) according to the CENELEC 50128 standard. The application of model-checking
to Ladder Logic programs in order to verify interlockings is well established within
academia and is beginning to see real applications in industry. As early as 1995, Groote
et al. [5] applied formal methods to verify an interlocking for controlling the Hoorn-Kersenbooger railway station. They conjecture the feasibility of verification techniques
as a mean of ensuring correctness criteria on larger railway yards. In 1998, Fokkink
and Hollingshead [6] suggested a systematic translation of Ladder Logic into Boolean
formulae. Newer approaches to interlocking verification have also been proposed in recent
years [7, 8, 9, 10, 11]. This includes work by Linh et al. which explores the verification
of interlockings written in a similar language to Ladder Logic using SAT-based model
checking. After two decades of research, academic work [12, 13] has shown that verification
approaches for Ladder Logic can indeed scale; in an industrial pilot, Duggan et al. [14]
conclude: “Formal proof as a means to verify safety has matured to the point where it
can be applied for any railway interlocking system.” In spite of this, such approaches
still lack widespread use within the UK Rail industry. Industrial standards for the railway and related domains increasingly rely on the application of formal methods for system analysis in order to establish a design’s correctness and robustness. Recent examples include the 2011 version of the CENELEC standard on railway applications, the 2011 ISO 26262 automotive standard, and the 2012 Formal Methods Supplement to the DO-178C standard for airborne systems. However, the application of formal methods research within the UK rail industry has yet to make a substantial impact. Principally our work aims to address one of the issues hindering its uptake, by removing the need for manual analysis of false negative error traces produced during verification.

Bounded model-checking is an efficient means of verifying a system through refutation. Given an abstracted model of a target system $M$ and some safety properties $\phi$, BMC searches for counterexamples up to some precomputed bound $k$. Search terminates when either an error trace is produced or the bound $k$ is reached. Determining this completeness threshold to sufficiently cover all states is often computationally intractable given it's true value will depend on model dynamics and size. Additionally, the k-induction rule, which checks if the property $\phi$ holds inductively for sequences of states, is constrained to acyclic graphs for guaranteeing completeness. 

An invariance rule allows us to establish an invariant property $\psi$ which holds for all initial states and transitions up to a given bound. Invariants may hold for sub-regions of the state space, meaning complete coverage isn't necessary to learn them. Supporting k-induction with strengthening invariants helps reduce the overall search space for bounded model checking, proving that invariant aspects of the system need not be considered. This can help filter cases where false negative counter examples are triggered by unreachable states. Generating sufficiently strong invariants is a non-trivial process given their construction is heavily program dependent. Usually such invariants require domain knowledge, typically devised by engineers responsible for the program implementation. 

We infer two principle challenges to address. First, formulating theoretical and practical frameworks in an academic-industry partnership with Siemens Rail Automation UK to represent interlocking verification as a goal-orientated reinforcement learning task. Second, devising an appropriate strategy for learning invariants within those frameworks.

The aim of this paper is to address this first challenge. Reinforcement learning is a popular machine learning paradigm with a demonstrably impressive capacity for learning near optimal strategies for goal-orientated tasks. Such approaches are well suited to problems where one need systematically learn the behaviour of a deterministic system, record observations regarding different 'states' of behaviour and identify patterns across those states. Generalisation of learned policies across different 'environments' being one of the key challenges in RL, we first aim to learn conceptually simpler goals, such as finding the progressively larger upper bounds for BMC. Understanding how a verification problem can be formulated as one of where machine learning can be successfully applied suggests the promise of learning invariants this way.


\section{Preliminaries}\label{sec:preliminaries}

\subsection{Ladder Logic}
Ladder logic programs used to implement interlocking systems allow engineers to define system behaviour in terms of conditions (rungs) and outputs (coils) expressed in a derivation of boolean algebra. Existing SAT-based approaches have already demonstrated translations from ladder logic programs to propositional formulae. The valuation of variables comprising those expressions at any given time is an instance of a program state. The set of reachable states then constitutes the abstractions notion of complete system behaviour.

We can further abstract the program's behaviour by considering any one unique valuation of all variables constitutes a unique program 'state'. Program diagrams are comprised of horizontal lines, referred
to as rungs, and logical connectors between symbols, known as coils and contacts. Two types of contacts are used to represent the value of a contained program variable, also referred to as a latch. Open contacts as shown in Fig. 2.2(a) represent the unchanged value of a latch while closed contacts denote their negated value, Fig. 2.2(b). Latches often represent physical inputs to the system but are also used for any other variables required. Coils can represent one of two output types. These are either physical outputs
for some system component or a computed value to be used later in the program. Output values are also referred to as latches and can be negated with a closed coil. Each rung is read from left to right and, using the logical connectives shown in Fig. 2.3, represent a condition or rule within the program. Rungs are closed off with a coil representing conditional output on the right hand side. Using coils and contacts enables the expression of logical propositions in our programs. 

\subsection{Finite State Automata}
Observing the equivalence relations above note that certain variables appear repeatedly
throughout the formula, primarily the latch crossing. Each rung computes a new coil
value to be used in the next execution cycle hence the significance of rung ordering. To illustrate this change over time we introduce a new form of abstract representation, the finite state automaton.

\begin{definition}[Finite State Automaton]
	A finite state automaton $\mathcal{M}$ given a ladder logic formula $\psi\mathcal{L}$ is denoted $\mathcal{M(\psi\L)}$ and represented by the 4-tuple $\langle Q, \sum, \sigma, I_s \rangle$, where :
	\begin{itemize}
		\item $Q$ 
		\item $\sum$
		\item $\sigma$
		\item $I_s$
	\end{itemize}
\end{definition}


\subsection{Bounded Model-Checking}
Model checking is a formal verification technique stemming from the need to systematically
check whether certain properties hold for different configurations (states) of a given system.
Given a finite transition system T and a formula F, model checking attempts to verify
that $s \vdash F$ for every system state $s \in T$, such that $T \vdash F$. Model checking is typically
performed over three distinct phases as described by Baier [20]. Initially the model, that is
the mathematical representation of the system, must be constructed. This is accomplished
using a model description language, of which there are several. Next the model must
be tested through simulation to ensure the abstracted system functions as expected.
Provided the constructed model behaves adequately the final stage of modelling requires
formulating the properties to be verified according to some specification language. The
constructed model and property definitions then enable the model checker to perform runs
of the system for state verification. The model checking process culminates in the analysis
phase wherein verification results are generated. Properties which hold for all tested states
produce a ‘safe’ output. In the event a state is found to violate any specified properties, that
is s 6|= F, a counter example trace is provided by the model checker indicating which state(s)
caused the violation. Results may also indicate that the model, property formulation or
simulation process are insufficient for verification and therefore require further refinement.
If so, the adjustments are made and the pipeline is revised and repeated.
Verification means abstraction -  "...existing works have successfully mapped ladder logic semantics to propositional logic. Here expressions can be reordered in CNF ready for a SAT-solver."

Abstraction means translating logic to a model - "...any unique valuation of the variables comprising the target program is viewed as an individual system state. This process is expanded in sec \ref{sec:preliminaries}"

Modelling means we can apply verification - "...decomposing the target system to an abstracted set of states allows model-checking to systematically check each state for safety conflicts"

Verification has limitations - "... while a systematic process there are no guarantees of complete coverage. The longest loop-free path would be useful knowledge"
Bounded model checking (BMC) provides a subset of the state space to avoid combin-
atorial explosion given the number of unreachable states. A bound is defined as a finite
number of transitions between states. However unless all states are reached within that
number of transitions then the exploration process is incomplete. This trade-off ensures
that unreachable states are not considered but does not guarantee completeness within
the bounded model

Primary limitations of the solution presented by Kanso are address in [16, 12, 17, 18].
Due to the inductive step of verification the tool checks to see if a given state satisfies
some condition but does not consider the possibility of unreachable states which violate
the same safety condition. Consequently if such benign violations are detected the tool
risks generating false negative counter examples which requires manual inspection by an
experienced engineer. This concept is illustrated in Fig. 3.5. A potential solution to this
problem was introduced in subsequent work by James in 2010 [17]. In their exploration of
SAT-based bounded model checking, James addresses the limitations of Kanso’s approach
in terms of state reachability by introducing invariants to suppress false negatives.

Clearly, introducing appropriate invariants to avoid unattainable configurations re-
quires a comprehensive understanding of both the abstracted and physical system.
Additionally, as system complexity increases so do the number of unreachable states. This
could potentially necessitate thousands of unique invariants to suppress triggered false
negatives. Currently this process requires manual analysis by specialist engineers making
automation highly desirable. Generating sufficiently strong invariants automatically is
an extremely complex task, one which has received considerable attention in academic
literature. From software engineering techniques [21, 22, 23, 24, 25, 26, 27] to hybrid
methods incorporating machine learning [28, 29, 30, 31, 32], researchers have proposed
various approaches to invariant finding with varying degrees of success. 

\subsection{Reinforcement Learning}
Reinforcement learning has shown recent success in automating and optimising sequential decision making problems with respect to some goal-orientated tasks. Such problems are commonly formulated as optimal control of incompletely-known MDPs \cite{sutton2018reinforcement}. 

RL differs from other data driven ML paradigms in its ability to automate tasks without the
dependency of existing datasets. Given an environment $\epsilon$, a software agent

- A machine learning paradigm that models goal-orientated tasks as an environment within which software agents can interact and explore. 

- Environments are constructed with MDPs. 
\subsubsection{Terminology}
\begin{itemize}
	\item $s, s^{\prime}$ - states
	\item $a$ - an action
	\item $r$ - a reward
	\item $S$ - the set of all non-terminal states
	\item $S^{+}$ - the set of all states, including terminal states
	\item $A(s)$ - set of all actions for state $s$
	\item $\mathcal{R}$ - set of all possible rewards, a finite subset of $\mathbb{R}$

\end{itemize}

- Agents interact with the environment according to a policy

- Actions produce reward signals

- Reward signals help evaluate policies by computing expected returns

- Evaluation methods are task dependent


Reinforcement learning (RL) is one such area of ML which leverages human ability to formulate problems and utilise computational means of solving them through a process of accelerated trial and error. RL centres around the concept of a software agent capable of learning optimal policies for interacting with a human specified environment, traditionally modelled as an MDP or its variants. 

Deep reinforcement learning (DRL) addresses the limited capacity of Q-functions in determining an optimal policy for complex problems. Instead a neural network, called the policy network, can be implemented to approximate the optimal Q-function. This process is known as deep Q-learning where our policy network is a deep Q-network (DQN) [46]. The DQN receives a number of states from the environment as input to approximate a function generating the optimal Q-value such that it satisfies the Bellman equation. Q-values output by the network are then compared to an approximated 'optimal' target value in order to compute the loss. Ideal for approximation, the deep neural network architecture will then learn to minimise this loss via stochastic gradient descent and back propagation. This process is repeated for all state-action pairs until the optimal Q-function has been approximated. 


To support the training of our policy network we generate a dataset of experiences
sampled at each time step $t$. A value $N$ is set to limit the size of our dataset, referred
to as replay memory. Using training experiences from replay memory helps avoid learning linear correlations over short step sequences. which less influential on learning. Our first training cycle involves setting a replay memory size N and random weight initialisation for the policy network. Any state preprocessing is then performed before being passed to the DQN. We compute the loss for the given Q-value approximated by the network for the action listed in our most recent sample from replay memory.
In order to compute the loss for the current training iteration, we require some method for
calculating the maximum expected discounted reward for the next state. Without a Q-table
storing a comparable value for s$^{\prime}$ we must approximate this value by passing s$^{\prime}$ to the policy network using different samples from replay memory. Thus we can determine the
greatest Q-value output by the network over all actions a$^{\prime}$1,...,a$^{\prime}$n, in s$^{\prime}$. Next we perform gradient descent, typically stochastic gradient descent (SGD), to minimise the loss and
update the weights of our DQN. We repeat this process for every new time step until an
optimal policy is found. Alternatively we incorporate a second DQN, called the target network to mitigate over-approximations of maximum future expected reward when using a single policy
network. Given weights are only adjusted once the loss has been computed, both $q(s,a)$
and $q(s^{\prime},a^{\prime})$ are approximated using the same weights. This introduces undesirable properties for learning as both sets of network parameters are coupled. As the network weights adjust, $q(s,a)$ approaches the target value while our approximation of $q^\star(s^{\prime}$,a$^{\prime}$) continues to change. Using a second DQN allows us to adjust a separate set of weights for outputting target Q-values. These are then
adjusted at arbitrary time steps to match the current weights of the policy network.

\begin{theorem}[Experience Replay]
	The agent’s experience at a given time step, et is
	defined:
	\begin{equation}
	e_{t} = (st,at,rt+1,st+1)
	\end{equation}
\end{theorem}

\subsection{MDPs}
The essential process behind RL is that of gamification [45, 46, 47]. Depending on the task we wish the agent to learn, our environment is encoded with a rule set which describes performable actions and a reward scheme capable of issuing signals which penalise or reinforce agent decisions. We treat this reward as an in-game score, of which the agent is aware. The action set available to the agent, paired with some state transition rules, yield subsequent environment states. Agents are then incentivised to explore their environments through sampling actions, storing observations regarding the environment.

Formally, we define the environment our agent must traverse as a sequential mathematical framework known
as a Markov decision process (MDP). 

Consequently, agents attempt to converge toward an optimal deterministic or stochastic policy which maximises cumulative future reward. This can be expressed probabilistically $P(s^{\prime},r|s,a)$. Given the current state $s$ and an action $a$ is performed, what is the probability of a subsequent state $s^{\prime}$ leading to a reward $r$. Rewards signals are often sparse, meaning agents may traverse several states before receiving any feedback. Thus future states must be considered at each time step $t$. Complex tasks oft comprise  What constitutes the terminal step depends on the environment and how it is implemented. In our case we require some metric of agent performance other than reward to determine whether the environment should reset. Such problems are referred to as episodic tasks.1. To this end we introduce a terminal condition which computes the total
number of unique transitions and states to check if the agent has discovered them all.
Throughout the training process it is expected the agent will minimise the number of
necessary transitions to the point an optimal path is learned. Continuous tasks, which
have no terminal step, face the challenge of enumerating future cumulative reward for
infinitely many steps. One solution to dealing with an infinite horizon is discounting.

\subsection{Policy Learning}
A policy is a rule set learned by an agent which determines the next best action. This is determined based on samples from the current probability distribution over any given action yielding a positive reward given the state.

\begin{definition}[Policy]A probability distribution $\pi(a|s)$ describes the likelihood of action (a) returning a positive reward given environment state (s).
\end{definition}

\begin{definition}[Trajectory]
	A sequence $\tau$, comprised of the states s0,...,sn, actions
	a0,...,an, and rewards r0,...,rn experienced by an agent following a policy $\pi$
\end{definition}
The start state s0 is randomly sampled from a start state distribution, Is in our
case. Continuous random sampling of the probability distribution is done to avoid
repetitive behaviour. Initially agents have no understanding of their environment or
which actions lead to the greatest cumulative positive reward. Most initial actions will
result in negative scoring as parameters adjust over time. Agents are expected to perform
poorly at first, particularly with larger action sets given the potential number of sequences
to sample. It is this evolutionary process that allows RL agents to generalise so well
when learning long term goals.
In the context of this project we have two options for representing our environment.
1) Using maximum coverage where the environment consists of all potential system
configurations or 2) As the subset of all reachable states produced by executions of the
pelican crossing program. The first environment consists of 4096 states, each with twelve
directed edges representing the change in value for a single variable. Thus our action
set comprises all propositional variables in V where a single action entails negating
one boolean. The second environment consists of six states, those reachable by the
system. Each state in this environment changes based on the value of input variable
pressed, resulting in an action set {0,1}.

\begin{theorem}
	(Optimal Policy): iven a finite MDP, there exists at least one optimal
	policy J, over a set of parameters $\theta$, such that:
	$J(\theta) = E\pi[r(\tau)]$
	Finding the optimal policy requires some measure of decision quality. These are
	referred to as value functions
\end{theorem}

\subsection{Value Functions}
Value functions return an expected reward, which determines the overall benefit of a given
policy. They can be differentiated by the observations they use to determine the value. So called state-value functions, denoted v$\pi$, provides the value assigned to a state under
policy $\pi$. Informally, state-value functions produce a valuation assigned to the current
state which determines the expected reward from subsequently adhering to the same
policy $\pi$. We define the value function v$\pi$(s):

% Equations 5.1, 5.2

\begin{theorem}
	(Optimal state-value function): An optimal policy J($\theta$) has an optimal
	state-value function for all s $\in$ S, defined:
	
	\begin{equation}
		v\star(s) = max\pi v\pi(s)
	\end{equation}
\end{theorem}
Informally the optimal state-value function returns the maximum possible expected
return of any policy for each state. Additionally action-value functions, denoted q$\pi$,
provides the value assigned to an action under policy $\pi$. Action-value functions, also
knowns as Q-functions, produce a quality valuation assigned to the current action which
determines the expected reward from subsequently adhering to the same policy $\pi$. We
define the value function q$\pi$(s):
 % Equations 5.4, 5.5
 
The output of our Q-function is known as the Q-value.

\begin{theorem}
An optimal policy, J$\theta$ has an optimal Q-function for
all s $\in$ S and all a $\in$A:
\begin{equation}
	Q\star(s,a) = max\pi Q\pi(s,a)
\end{equation} 
\end{theorem}

Informally, the optimal Q-function returns the maximum possible expected return
of any policy for all state-action pairs. The Bellman principle of optimality [48], Eq. 5.7,
states for any state-action pair at the current time step, the expected return from an initial
state s, taking action a and following the optimal policy J($\theta$) thereafter is equal to the
expected reward from taking action a in state s, plus the maximum achievable expected
discounted return from any subsequent state-action pairs.

The Bellman optimality equation is an integral metric used to learn the optimal Q-function
which in turn is used to learn the optimal policy. Given an optimal Q-function action a$^{prime}$, a
Q-learning algorithm will find the best action a$^{\prime}$ which maximises the Q-value for s$^{\prime}$.

% Equation 5.7

\subsection{Reward/Goal Shaping}
Q-learning refers to a policy learning method which uses Q-functions to calculate maximum
expected future reward. Learning is formulated as an iterative process of parameter
adjustment known as value iteration.

\subsection{Exploration Strategy}
In the previous section we briefly discussed initial exploration rates to compensate for
zeroed Q-tables. A popular approach to setting exploration rates is the epsilon greedy
strategy [49, 50, 51, 52]. In order to balance the exploration-exploitation trade-off [53, 54],
we set an exploration rate $\epsilon$ between 0 and 1 to dictate whether the agent prioritises
exploratory or exploitative behaviour. Values nearing 0 represent a greedy strategy where
the agent is more likely to exploit previous knowledge. Epsilon values nearing 1 therefore
encourage exploratory behaviour. Additionally we assign an epsilon decay value to
decrease the exploration rate for each episode. Including an epsilon decay value provides
some beneficial learning properties. As the agent learns more about the environment it
relies less on exploration. Instead we expect to have observed a sufficiently comprehensive
understanding of the environment thus settle on a purely greedy strategy. We also
influence exploration and exploitation at every time step. First we randomly generate
a number, i between 0 and 1 to compare with the current value of $\epsilon$. If i > $\epsilon$ the agent
selects an action with the highest Q-value for the state-action pair. Where i < $\epsilon$, the agent
randomly samples an action to explore the environment.
In this project we aim to train an RL agent to learn the optimal path through our
pelican crossing state space. Say we issue a positive reward of 1 for every new state
discovered and -1 for repeated transitions. To update the new Q-value we approximate
the right hand ride of the Bellman equation. First, we iteratively compare the loss of
the current Q-value and the optimal Q-value for each state-action pair. Our aim is to
minimise this loss until convergence.

% Equations 5.8, 5.9

Designing appropriate reward systems is arguably one of the greatest challenges in RL. Reward shaping is often performed specifically for the application

% Definition 5.10

\section{Mapping Formal Methods to RL}
Machine learning means translating the model to another model - "...Which requires some partial translation, process is expanded in sect \ref{sec:preliminaries}".

\section{Implementation}
\subsection{LL program generator}
\subsection{Learning Environment}
Constructing an environment is arguable the longest phase of training an RL agent. To
ensure our agent performs well in real-world applications, the learning environment must
be sufficiently representative of the problem domain. Through what is essentially the
process of gamification, we design an environment that records agent actions and issues a
reward based on the behaviour we wish to reinforce. This process presents a number of
distinct challenges in our context. First, given a simple ladder logic program, is it possible
to encode its functionality and constraints using an imperative programming language.
Second, an efficient method of indexing visited states during exploration to determine a
k-number of steps before the agent is forced to revisit states. Finally, we must identify the
optimal reward scheme and learning parameters to enforce the desired behaviour. We design an openai-gym [102] like environment to train our agent (Appendix A).
Consider the environment as a game. The agent makes successive transitions between
states from some arbitrary start position sampled from Is. Recall the set I  defined our physical inputs received by the ladder logic program shown in sect. 3.2. At each step the
agent is presented with a binary decision to make, selected from our action set A, where: $A \equiv I$

Our single input variable pressed is valued either 0 or 1. Once an action is selected, each
transition must then be computed as a single execution of the ladder logic program. We
define a transition function, Alg. 1, which receives an action ai, and the current state as
input. Depending on whether the agent selects 0 or 1, the function returns a new state
with an updated valuation. These are the transitions $\sigma$ : q $\sum$→ q$^{\prime}$ defined in sect 3.3 w.r.t
finite state automata. Under the assignment rules of our transition function, the agent will
eventually discover all six reachable states. Fig. 8.1 represents the finished environment
with all paths the agent can explore. Note this is essentially a more complete version of
our automaton in sect. 3.3. The formulation of our environment means the agent benefits
from having to traverse a directed graph. This can often reduce the number of decisions to
consider at each step particularly for large graphs with many edges. Additionally our agent
samples from a small action set, meaning fewer potential transitions from a single node. With our transition function generating a total of six states, we continue to implement
a collection of traces to record agent exploration. With each action we store the ensuing
transition, a 3-tuple (q,$a_{i}$,q$^{\prime}$), and the new state q$^{\prime}$ in separate hash tables. Our traces let us determine whether the current state or latest transition is unique given we have
a record of past steps.



\subsection{Agent Training}
Given we wish to find the maximal depth of the state space without repeating transitions,
a positive reward is issued for new discoveries. Contrarily we issue a negative reward
for adding previously recorded states with a harsher reward for repeating transitions. In
regard to the exploration-exploitation trade-off we issue a larger reward for discovering
new transitions over new states. This is primarily due to agent’s prioritising actions that
are guaranteed to issue positive reward while attempting to avoid negative scores. To avoid
infinite loops we introduce a terminal condition which resets the environment when all
states and transitions have been discovered. To encourage the agent to satisfy the terminal
condition as soon as possible we issue a large positive reward once it has been reached.

We can use
DQN(s) to approximate the optimal Q-function and, consequently the optimal policy. o this end we implement two DQNs, the first policy network to train our agent
and the second target network to approximate our target values. DQNs train differently
to Q-tables in their use of replay memory. We implement a basic Replay class to store
sequences of 5-tuples (s,a,s$^{\prime}$,r), where s is the current state, a denotes the action taken,
s$^{\prime}$ representing the next state and r being the reward. We then define a max capacity N,
which dictates the number of experiences our agent remembers. One the max capacity
is reached, the agent pops the oldest entry and adds the latest experience to memory.
During the training process our agent samples experiences according to some arbitrary
batch size. If the replay memory is less than this batch size, we cannot sample. Therefore
we introduce some basic condition checking to determine whether sampling is feasible at
any given time step. In regard to the selected exploration strategy, $\epsilon$, we employ the same
implementation used for Q-tables. Here we set an initial value for $\epsilon$, a minimum value
and our $\epsilon$-decay rate. Concerning our DQN architecture, we construct an input layer of 12
nodes, one for each $v_{i} \in$ V , two fully connected hidden layers and a binary output layer to
reflect the action set A. To avoid the problem of moving targets, we set an update rate of 10
epochs for our target network. That is to say we update the weights of the target network for every 10 passes through our policy network. Our chosen optimiser is Adam [103], an
SGD variant based on AdaGrad [104] and RMSProp [105], due to its speed and accuracy.
\subsubsection{Network Architecture}

\section{Experiments}
\subsection{Extended state-spaces}
The agent learns over time to avoid repeated actions until an optimal trace is output.
This trace can then be used to find the maximal depth of the state space without looping
for all states in Is, if such a path exists. We discover through a series of experiments
that there are two sets of unique transitions for all start states such that no transition is
repeated twice. Concerning the discovery of loop free paths, as discussed in [17], it may
be possible to either explicitly state which path should be verified by the SAT-solver or
leverage discoveries made by the agent in identifying a loop free path. By increasing the
max step threshold within our environment allows the agent sufficient time to explore the
state space and understand what satisfies the terminal condition. Consequently, provided
a long enough training time, the agent will always find an optimal path. In fact it was
discovered through a series of experiments, specifically ones with a low $\epsilon$-decay rate, that
there exist two optimal paths for every start state. These traces can then be analysed to
determine the maximum k-steps before looping.
\subsection{Interlocking examples}





\begin{table}
\caption{Table captions should be placed above the
tables.}\label{tab1}
\begin{tabular}{|l|l|l|}
\hline
Heading level &  Example & Font size and style\\
\hline
Title (centered) &  {\Large\bfseries Lecture Notes} & 14 point, bold\\
1st-level heading &  {\large\bfseries 1 Introduction} & 12 point, bold\\
2nd-level heading & {\bfseries 2.1 Printing Area} & 10 point, bold\\
3rd-level heading & {\bfseries Run-in Heading in Bold.} Text follows & 10 point, bold\\
4th-level heading & {\itshape Lowest Level Heading.} Text follows & 10 point, italic\\
\hline
\end{tabular}
\end{table}


\noindent Displayed equations are centered and set on a separate
line.
\begin{equation}
x + y = z
\end{equation}
Please try to avoid rasterized images for line-art diagrams and
schemas. Whenever possible, use vector graphics instead (see
Fig.~\ref{fig1}).


\begin{theorem}
This is a sample theorem. The run-in heading is set in bold, while
the following text appears in italics. Definitions, lemmas,
propositions, and corollaries are styled the same way.
\end{theorem}
%
% the environments 'definition', 'lemma', 'proposition', 'corollary',
% 'remark', and 'example' are defined in the LLNCS documentclass as well.
%
\begin{proof}
Proofs, examples, and remarks have the initial word in italics,
while the following text appears in normal font.
\end{proof}
For citations of references, we prefer the use of square brackets
and consecutive numbers. Citations using labels or the author/year
convention are also acceptable. The following bibliography provides
a sample reference list with entries for journal
articles~\cite{ref_article1}, an LNCS chapter~\cite{ref_lncs1}, a
book~\cite{ref_book1}, proceedings without editors~\cite{ref_proc1},
and a homepage~\cite{ref_url1}. Multiple citations are grouped
\cite{ref_article1,ref_lncs1,ref_book1},
\cite{ref_article1,ref_book1,ref_proc1,ref_url1}.
%
% ---- Bibliography ----
%
% BibTeX users should specify bibliography style 'splncs04'.
\bibliographystyle{splncs04}
\bibliography{bibliography}
% References will then be sorted and formatted in the correct style.
%
% \bibliographystyle{splncs04}
% \bibliography{mybibliography}
%

\end{document}
